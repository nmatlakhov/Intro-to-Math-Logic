\documentclass[12pt]{article}
\usepackage[left=1cm, right=1cm, top=2cm,bottom=1.5cm]{geometry} 

\usepackage[parfill]{parskip}
\usepackage[utf8]{inputenc}
\usepackage[T2A]{fontenc}
\usepackage[russian]{babel}
\usepackage{enumitem}
\usepackage[normalem]{ulem}
\usepackage{amsfonts, amsmath, amsthm, amssymb, mathtools}

\usepackage{fancyhdr}
\pagestyle{fancy}
\renewcommand{\headrulewidth}{1.5pt}
\renewcommand{\footrulewidth}{1pt}

\usepackage{graphicx}
\usepackage[figurename=Рис.]{caption}
\usepackage{subcaption}
\usepackage{float}

%%Наименование папки откуда забирать изображения
\graphicspath{ {./images/} }

%%Изменение формата для ввода доказательства
\renewcommand{\proofname}{$\square$  \nopunct}
\renewcommand\qedsymbol{$\blacksquare$}

\addto\captionsrussian{%
	\renewcommand{\proofname}{$\square$ \nopunct}%
}
%% Римские цифры
\newcommand{\RN}[1]{%
	\textup{\uppercase\expandafter{\romannumeral#1}}%
}


\theoremstyle{definition}
\newtheorem{defn}{Опр:}
\newtheorem{rem}{Rm:}
\newtheorem{prop}{Утв.}
\newtheorem{exrc}{Упр.}
\newtheorem{lemma}{Лемма}
\newtheorem{theorem}{Теорема}
\newtheorem{corollary}{Следствие}

\newenvironment{cusdefn}[1]
{\renewcommand\thedefn{#1}\defn}
{\enddefn}



\DeclareRobustCommand{\divby}{%
	\mathrel{\text{\vbox{\baselineskip.65ex\lineskiplimit0pt\hbox{.}\hbox{.}\hbox{.}}}}%
}


\newcommand{\smallerrel}[1]{\mathrel{\mathpalette\smallerrelaux{#1}}}
\newcommand{\smallerrelaux}[2]{\raisebox{.1ex}{\scalebox{.75}{$#1#2$}}}

\newcommand{\smallin}{\smallerrel{\in}}
\newcommand{\smallnotin}{\smallerrel{\notin}}


\begin{document}
	\lhead{Математическая логика}
	\chead{Беклимишев Л.Д.}
	\rhead{Лекция - 1}
	
\subsection*{План курса:}	
\begin{enumerate}
	\item Теория множеств.
	\item Логика высказываний.
	\item Логика предикатов.
	\item Вычислимость. Теорема Геделя.
\end{enumerate}

\section*{Аксиоматика теории множеств}
\uwave{Синтаксис} - грамматика формального языка.\\
\uwave{Семантика} - каким образом наделяются смыслом формальные высказывания.

\underline{Парадокс рассела} (парадокс брадобрея):

$R = \{\,x \mid x \notin x\,\}$, $x \in R \Leftrightarrow x \notin x$, $R \in R \Leftrightarrow R \notin R$

Э. Цермело, А. Френкель - построили аксиоматическую теорию множеств (ZF).

\subsection*{Аксиоматический метод:}

\begin{enumerate}[label={(\arabic*)}]
	\item Зафиксировать базисные, \underline{неопределимые понятия}.
	\item Постулируем некоторое число аксиом, связывающих между собой базисные понятия.
	\item Выводим следствия по \underline{правилам логики}.
\end{enumerate}

\uwave{Аксиоматика Тарского} имеет неопределимые понятия: (1) точка, (2) отношение между тремя точками, (3) отношение между четырьмя точками = длины отрезков, (4) равенство точек.

\uwave{Аксиоматика Дедекинда, Пеано} имеет неопределимые понятия: (1) точка 0, (2) операция +1/функция $S$ и следующие аксиомы:

\begin{enumerate}[label={(\arabic*)}]
	\item $\forall n, \neg S(n) = 0$
	\item $S(n) = S(m) \Leftrightarrow n = m$
	\item $P(0) \wedge \forall n (P(n) \Rightarrow P(n+1)) \Rightarrow \forall n, P(n)$, $P$ - любое свойство.
\end{enumerate}

Что такое свойство мы пока не знаем.

К языку арифметики добавим простые аксиомы: $n + 0 = n$, $n + S(m) = S(n+m) \Rightarrow$ получим арифметику Пеано $\Rightarrow$ свойство = некоторое высказывание о натуральных числах.

$x, X$ - множества - некая совокупность элементов, объединенная каким-то свойством. \\
$\varnothing$ - пустое множество. 

Нужно заметить, что $x \neq \{x\}$, $\{x\} \neq \{\{x\}\}$. $x$ - в нем может быть множество элементов, $\{x\}$ - здесь один элемент.

\uwave{Неопределимые понятия}: (1) множества = любые объекты, (2) принадлежность множеству $x \in y$.

Иногда определяют равенство множеств, как аксиому, мы рассматриваем это как опредление, в частности, по определению в множестве нет повторяющихся элементов.


\begin{defn}
	$X = Y \overset{\text{def}}{\Longleftrightarrow} \forall z, (\, z\in X \Leftrightarrow z \in Y \,)$ (Аксиома объемности).
\end{defn}

\begin{defn}
	$X \subseteq Y \overset{\text{def}}{\Longleftrightarrow} \forall z, (\, z\in X \Rightarrow z \in Y \,)$; $X \subsetneq Y \overset{\text{def}}{\Longleftrightarrow} (X \subseteq Y \wedge X \neq Y)$
\end{defn}


\textbf{(1) Аксиома равенства:} $x = y \Rightarrow \forall z,  (x\in z \Leftrightarrow y \in z)$\\
Отсюда следует, что $\exists$ хотя бы 1 элемент: $\exists x\colon x = x$ - ``аксиома существования'' - далее будет выводиться в кванторах.

\textbf{(2) Аксиома пары:} $\forall x, y \; \exists z = \{x,y\} \colon \forall u, \big(u\in z \Leftrightarrow (u = x \vee u = y)\big)$

\begin{defn}
	$\{x\}\stackrel{\text{def}}{=} \{x,x\}$ - \uwave{синглетон}. Синглетон существует по аксиоме пары.
\end{defn}

$x \in \{x\}$, $\mathbb{R}$ - множество элементов, $\{\mathbb{R}\}$ - один элемент.

\begin{proof}
		$\{x\} = \{x,x\} \Leftrightarrow \forall z, (z \in \{x\} \Leftrightarrow z \in \{x,x\})$, по определения равенства.\\ 
		$z \in  \{x\} \Leftrightarrow (z =x)$, $z \in \{x,x\} \Leftrightarrow (z = x \vee z =x) \Leftrightarrow z = x$.\\ 
		Строго можно рассамтривать $\{x\} = \{z \mid z = x\}$.
\end{proof}

\begin{prop}
	$x = y \Leftrightarrow \{x\} = \{y\}$.
\end{prop}

\begin{proof}
	$x = y \Leftrightarrow \forall z \; (z = x \Leftrightarrow z = y) \Leftrightarrow \forall z (z\in \{x\} \Leftrightarrow z \in \{y\}) \Leftrightarrow \{x\} = \{y\}$.
\end{proof}

\textbf{(3) Аксиома объединения:} $\forall x \; \exists y = \cup x \colon \forall u, (u \in y \Leftrightarrow \exists z \; (u \in z \wedge z \in x))$.

$\varnothing$ - множество у которого нет ни одного элемента, то есть $\forall x \; x \notin \varnothing$, сможем получить его после 5-ой аксиомы.

\begin{prop}
	Множество $\varnothing$ - единственно.
\end{prop}

\begin{proof}
	Пусть $\forall x \; x \notin \varnothing, \forall x \; x \notin \varnothing^\prime \Rightarrow \forall x \; (x \notin \varnothing \Leftrightarrow x \notin \varnothing^\prime) \Rightarrow \forall x \; (x \in \varnothing \Leftrightarrow x \in \varnothing^\prime) \Leftrightarrow \varnothing = \varnothing^\prime$
\end{proof}

$X = \Big\{\{\varnothing\}, \big\{\{\varnothing\}\big\}, \big\{\varnothing, \{\varnothing\}\big\}\Big\}$ - содержит 3 элемента, все сущуствуют по аксиоме пары.

$\bigcup X = \big\{\varnothing,\{\varnothing\}, \varnothing,\{\varnothing\} \big\} = \big\{\varnothing,\{\varnothing\} \big\}$
	
$\{x, y, z\} = \{x, y\} \cup \{z\} = \bigcup \big\{\{x,y\}, \{z\}\big\}$ - аналогично для более широких множеств.

$X \cup Y \overset{\text{def}}{\coloneqq} \bigcup \{X, Y\}$
	
\textbf{(4) Аксиома степени:} $\forall x \; \exists y = \mathcal{P}(x) \colon \; \forall z \; (z \in y \Leftrightarrow z \subseteq x)$, где $\mathcal{P}(x)$ - множество всех подмножеств.

\textbf{(5) Аксиома выделения:} $\forall x \; \exists y = \{u \in x \mid \varphi(u)\} \colon \forall u \; \Big(u \in y \Leftrightarrow \big( u \in x \wedge \varphi(u)\big)\Big)$, где $\varphi$ - это какое-то свойство, некоторый признак элемента $u$.

$A \rightsquigarrow \{x \in A \mid \varphi(x)\}$ - множество всех элементов A, обладающих свойством $\varphi$.

\subsection*{Примеры:}

$X \subseteq \mathbb{N}$, $X = \{n \in \mathbb{N} \mid n \text{ - нечетно}\}$ - использовали аксиому выделения.

$Y \subseteq \mathbb{R}$, $Y = \{x \in \mathbb{R} \mid x < a\}$ - использовали аксиому выделения.

\subsection*{Множества и классы:}
Множество всех множеств - не употребляется, говорят \underline{класс множеств}.

\begin{defn}
	\uwave{Класс}$ = \{x \mid \varphi(x)\}$ - совокупность всех $x$, удовлетворяющих $\varphi$, которые могут быть или не быть самими множествами.
\end{defn}

\begin{prop}
	$\exists \varnothing$ множество.
\end{prop}

\begin{proof}
	$\exists x_\circ$ (по аксиоме равенства/аксиоме существования) $\Rightarrow$ по аксиоме выделения\\ $\exists \{x \in x_\circ \mid x \notin x_\circ\} = \varnothing = \{x \in x_\circ \mid x \neq x\}$
\end{proof}


$\langle x,y\rangle$ - упорядоченная пара.
\begin{prop}
	Основное свойство упорядоченной пары: $\langle x_1,y_1\rangle = \langle x_2, y_2 \rangle \Leftrightarrow (x_1 = x_2 \wedge y_1 = y_2)$.	
\end{prop}

\begin{proof}
$(\Rightarrow) \langle x_1, y_1 \rangle =  \langle x_2, y_2 \rangle \Rightarrow \cap \langle x_1, y_1 \rangle = \cap \langle x_2, y_2 \rangle \Rightarrow \{x_1\} = \{x_2\} \Leftrightarrow x_1 = x_2$\\
$\cup \langle x_1, y_1 \rangle = \cup \langle x_2, y_2 \rangle \Rightarrow \{y_1\} = \{y_2\} \Leftrightarrow y_1 = y_2$

$(\Leftarrow)$ верно по аксиоме равенства: $x_1 = x_2 \wedge y_1 = y_2 \Rightarrow$ любые построенные из них множества будут равны.
\end{proof}

\begin{defn}
	\uwave{Упорядоченная пара по Куратовскому:} $\langle x,y\rangle \overset{\text{def}}{\coloneqq} \big\{\{x,y\}, \{x\} \big\}$
\end{defn}

$y \in \cap x \Leftrightarrow (\forall u \in x, y \in u)$

\begin{theorem}
	$x \neq \varnothing \Rightarrow \exists! z \colon z = \cap x$
\end{theorem}

\begin{proof}
	$x \neq \varnothing \Rightarrow \exists a, a \in x \Rightarrow$ по (А5) в $a \; \exists z = \{y\in a \mid \forall u \in x, y \in u\} = \cap x$ (существование).\\
	$y \in z \Leftrightarrow (\forall u \in x, y \in u) \Leftrightarrow y \in z^\prime \Rightarrow$ по аксиоме объемности $\Rightarrow z = z^\prime$ (единственность).
\end{proof}

Как можно получить элементы из пары?

$\{x\} = \{x,y\} \cap \{x\} = \cap \langle x, y\rangle$

$\{y\}= \left\{ \begin{aligned}
	\big(\cup \langle x,y \rangle\big) \setminus \big(\cap \langle x,y \rangle\big),&& x \neq y, & \text{ разность множеств существует по аксиоме выделения}\\
	\cap \langle x,y \rangle, && \text{иначе} &  
\end{aligned}
\right.$


\end{document}